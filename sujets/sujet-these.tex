\documentclass[a4paper,11pt]{article}
\usepackage[french]{babel}
\usepackage[utf8]{inputenc}
\usepackage{hyperref}

\title{Théories des points-fixes et des coalgèbres appliquées à la coinduction en Coq}



\begin{document}

\maketitle

\section{Contexte}

La théorie des \emph{points fixes}~\cite{BIR67a}, et plus généralement des ordres partiels~\cite{DaveyPriestley90}, est une théorie bien développée, et très utilisée en informatique: algorithmique, logique, sémantique, interprétation abstraite, optimisation.

L'étude des \emph{coalgèbres}~\cite{jacobs:book} est une branche plus récente de la théorie des catégories, qui permet de modéliser toutes sortes de systèmes à états (automates, flots, processus concurrents, systèmes réactifs), et d'obtenir des outils pour raisonner sur ces systèmes (bisimulations, déterminisation).
%
La théorie des coalgèbres est en fait strictement duale de celle des algèbres, qui est souvent utilisée comme base du raisonnement inductif et équationnel. D'un point de vue programmation, algèbres initiales et coalgèbres finales correspondent naturellement aux notions de types inductifs (entiers, listes, arbres finis) et coinductifs (flots, listes et arbres potentiellements infinis).
%
Certains théorèmes sur les (co)algèbres peuvent être vus commes des généralisations de théorèmes de point fixe. Par exemple le théorème assurant l'existence d'une algèbre initial pour tout foncteur accessible~\cite{adamek1974free} généralise le théorème de Knaster-Tarski~\cite{Kna28,Tarski55} assurant l'existence d'un plus petit point fixe pour toute fonction monotone sur un treillis complet.

Enfin, les \emph{assistants de preuve} sont des outils permettant de formaliser et vérifier des résultats mathématiques (théorème des quatre couleurs, conjecture de Kepler) et/ou informatiques (compilateur certifié Compcert, micro noyau seL4). Ceux basés sur la théorie des types, tels que Agda, Coq, ou Lean, s'appuyent sur un language de programmation fortement typé et exploitent l'isomorphisme de Curry-Howard pour représenter les preuves par des programmes.

\section{Problématique}

TODO

Pas de general-purpose librairie de partial-order theory

% Simplified and specialised versions of these results have sometimes been formalised for the need of concrete projects. But a reference general-purpose library on those concepts is still missing, that would contain once and for all the key theorems under their most general form, in order to enable their reuse within many contexts.

Coinductifs pénibles en Coq

Théorème de point fixe sans contre-partie catégorique, 

\section{Directions de recherche}

TODO

a challenge consists in finding how to organise the library in a modular way, to deal uniformly with properties of partial orders with more or less structure (lattices, distributive lattices, complete lattices, cpos, pointed cpos, etc). In order to test the library, we will use it to factor the code of certain existing libraries for which ad-hoc formalisations of partial order theory had been required [4,5].

comment se passer du companion quand on ne sait pas s'il existe?

\section{Encadrement}

Cette thèse sera codirigée par Yannick Zakowski et Damien Pous, chercheurs des équipe CASH et Plume du LIP, à l'ENS de Lyon.

\bibliographystyle{abbrvurl}
\bibliography{refs}


\end{document}
