\documentclass[a4paper,11pt]{article}
\usepackage[french]{babel}
\usepackage[utf8]{inputenc}
\usepackage{hyperref}

\title{Théories des points-fixes et des coalgèbres appliquées à la corécursion en Coq}



\begin{document}

\maketitle

\section{Contexte}

La théorie des \emph{points fixes}~\cite{BIR67a}, et plus généralement des ordres partiels~\cite{DaveyPriestley90}, est une théorie bien développée, et très utilisée en informatique: algorithmique, logique, sémantique, interprétation abstraite, optimisation.

L'étude des \emph{coalgèbres}~\cite{jacobs:book} est une branche plus récente de la théorie des catégories, qui permet de modéliser toutes sortes de systèmes à états (automates, flots, processus concurrents, systèmes réactifs), et d'obtenir des outils pour raisonner sur ces systèmes (bisimulations, déterminisation).
%
La théorie des coalgèbres est en fait strictement duale de celle des algèbres, qui est souvent utilisée comme base du raisonnement inductif et équationnel. D'un point de vue programmation, algèbres initiales et coalgèbres finales correspondent naturellement aux notions de types inductifs (entiers, listes, arbres finis) et coinductifs (flots, listes et arbres potentiellement infinis).
%
Certains théorèmes sur les (co)algèbres peuvent être vus comme des généralisations de théorèmes de point fixe. Par exemple le théorème assurant l'existence d'une algèbre initiale pour tout foncteur accessible~\cite{adamek1974free} généralise le théorème de Knaster-Tarski~\cite{Kna28,Tarski55} assurant l'existence d'un plus petit point fixe pour toute fonction monotone sur un treillis complet.

Enfin, les \emph{assistants de preuve} sont des outils permettant de formaliser et vérifier des résultats mathématiques (théorème des quatre couleurs, conjecture de Kepler) et/ou informatiques (compilateur certifié Compcert, micro noyau seL4). Ceux basés sur la théorie des types, tels que Agda, Coq, ou Lean, s'appuient sur un langage de programmation fortement typé et exploitent l'isomorphisme de Curry-Howard pour représenter les preuves par des programmes.

\section{Problématiques}

Il reste de nombreuses questions ouvertes dans le domaine des coalgèbres. En particulier, 
certains théorèmes de point fixe n'ont pas encore de contrepartie claire dans ce cadre.
C'est le cas de certains résultats relativement avancés concernant les \emph{techniques up-to} pour la coinduction\cite{pous:lics16:cawu,pr19:lmcs:companion}, mais aussi de résultats de base comme le second théorème de point fixe de \cite[chapitre~8]{DaveyPriestley90}. 

Spécifiquement dans le cadre des assistants de preuves, et en particulier
l'assistant de preuve Coq, deux problèmes majeurs existent :
\begin{enumerate}
\item Alors que des parties de la théorie des ordres partiels et des versions spécialisées des théorèmes de points fixes ont déjà été formalisées pour les besoins de projets spécifiques, il n'y a toujours pas à ce jour de librairie couvrant une fois pour toutes ces outils standard, de manière robuste et modulaire.
\item Les outils proposés pour travailler avec des types coinductifs ne sont pas satisfaisants: d'une part, la notion native d'égalité n'est pas celle que l'on souhaite avoir sur les habitants de ces types, et d'autre part la définition de fonctions \emph{corécursives} repose sur une condition de garde ad-hoc pour garantir la correction du système, qui est source d'un bogue relativement sévère dans la théorie de Coq, et malgré tout très souvent contraignante pour le programmeur.
\end{enumerate}
Ces deux limitations sont problématiques notamment dans le domaine de la compilation certifiée, où l'on utilise souvent la théorie des domaines (et donc des points fixes), des connexions de Galois (idem), et où des types coinductifs apparaissent naturellement~\cite{compcert,itrees}.

Vis à vis de la seconde problématique, il peut paraître étonnant que la dualité que l'on observe au niveau catégorique entre structures inductives et structures coinductives ne se retrouve pas en théorie des types: les types inductifs s'y comportent très bien et sont bien compris, mais pas les types coinductifs.
Cet aspect semble être spécifique à la théorie des types: en Isabelle/HOL, qui repose sur la logique  d'ordre supérieur (une logique classique), de très bons outils ont pu être développés pour les types inductifs et coinductifs, en s'appuyant sur la théorie catégorique sous-jacente~\cite{DBLP:conf/frocos/BiendarraBBDFHK17}.

\section{Programme de recherche}

Nous proposons dans cette thèse d'étendre la théorie catégorique des coalgèbres afin de pouvoir l'implémenter en théorie des types (i.e., Coq) et fournir un meilleur support pour les types coinductifs.

Un tel programme comporte des questions à la fois mathématiques et de développement logiciel.
\begin{itemize}
\item Du côté mathématique se posent des questions de théorie des catégories (généralisation de certains théorèmes de point fixe, techniques up-to pour les coalgèbres) et de théorie des types (comment formuler certains énoncés pour qu'ils soient démontrables constructivement, en théorie des types).
\item Du côté développement logiciel se posent des questions de structuration: comment organiser les librairies développées de manière modulaire et réutilisable, comment partager le code et les preuves pour diverses variantes de certains résultats. 
\end{itemize}

Durant une première partie de la thèse, nous nous concentrerons sur la
formalisation de la théorie des ordres partiels et des points fixes. En effet,
cet objectif plus modeste contient déjà une partie des difficultés que l'on
s'attend à rencontrer avec notre objectif principal (preuves à rendre
constructives, modularité et partage du code dans la formalisation). Il nous
permettra d'anticiper d'éventuelles difficultés avec notre objectif principal,
et fournira un premier jalon déjà fort utile pour la communauté---addressant la
première limitation décrite en section précédente.

\section{Encadrement}

Cette thèse sera codirigée par Yannick Zakowski et Olivier Laurent, chercheurs des équipes CASH et Plume du LIP, à l'ENS de Lyon.

\bibliographystyle{abbrvurl}
\bibliography{refs,pous}


\end{document}
