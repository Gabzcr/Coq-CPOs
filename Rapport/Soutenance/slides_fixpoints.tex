\documentclass{beamer}

\usepackage[french]{babel}
\usepackage[utf8]{inputenc}
\usepackage[T1]{fontenc}
\usepackage{lmodern}

\usepackage{bbold}

\usepackage{subcaption}
\usepackage{listingsutf8}
\usepackage[french,onelanguage,ruled]{algorithm2e}

\definecolor{ltblue}{rgb}{0,0.4,0.4}
\definecolor{dkblue}{rgb}{0,0.1,0.6}
\definecolor{dkgreen}{rgb}{0,0.35,0}
\definecolor{dkgreen}{rgb}{0,0.35,0}
\definecolor{dkviolet}{rgb}{0.3,0,0.5}
\definecolor{dkpink}{rgb}{0.5,0,0.3}
\definecolor{dkred}{rgb}{0.5,0,0}
\definecolor{orange}{rgb}{0.9,0.5,0.3}
\definecolor{violet}{rgb}{0.7,0,0.7}

\usepackage{listings}
\usepackage{lstcoq}

%\lstset{escapeinside=||}

\usepackage{tikz}
\usetikzlibrary{arrows.meta}
\usetikzlibrary{shapes}

\uselanguage{French}
\languagepath{French}

\usetheme{Warsaw}

\usefonttheme[onlymath]{serif}


\title[\textit{Partial orders and fixpoint theorems, in Coq} ~~~~~~~~~~~~~~~~~~~~~ \insertframenumber]{Rapport de stage M2 \\
\textit{Partial orders and fixpoint theorems, in Coq}}
\date{février-juin 2022}
\author[Gabrielle Pauvert]{Gabrielle Pauvert\\
stage encadré par Damien Pous et Yannick Zakowski}


\setbeamertemplate{blocks}[rounded][shadow=true]
\setbeamertemplate{navigation symbols}{}

\newcommand\code[1]{{\fontfamily{lmtt}\selectfont #1}}
\newcommand\respect[2]{~(#1 ~\Longrightarrow~ #2)~}


\begin{document}

\begin{frame}
\titlepage
\end{frame}


%\section*{Introduction}

\begin{frame}
\frametitle{Introduction}

\begin{block}{Objectif du stage}
\begin{itemize}
\item[->] Développer une bibliothèque générale et modulaire pour Coq sur la théorie des ordres partiels et leurs principaux résultats de points fixes.
\end{itemize}
\end{block}

\begin{exampleblock}{Enjeux et utilités}
\begin{itemize}
\item regrouper les résultats utilisés dans d'autres projets (coinduction, algèbres relationnelles)
\item sémantique : donner du sens aux boucles while et aux récursions (points fixes de programmes)
\item logique, optimisation, interprétation abstraite, algorithmie, théorie des catégories etc.
\end{itemize}
\end{exampleblock}

\end{frame}


\begin{frame}
\frametitle{Sommaire}
\small{\tableofcontents}
\end{frame}


\AtBeginSection[]{
  \begin{frame}
  \vfill
  \centering
  \begin{beamercolorbox}[sep=8pt,center,shadow=true,rounded=true]{title}
    \usebeamerfont{title}\insertsectionhead\par
  \end{beamercolorbox}
  \vfill
  \end{frame}
}



\section{Ordres partiels et points fixes}


\subsection{Notions de théorie des ordres partiels}

\begin{frame}
\frametitle{Quelques notions de théorie des ordres partiels}

\begin{block}{Définitions}
\begin{itemize}
\item<1-> \textbf{Ordre Partiel} : relation binaire réflexive, transitive et antisymétrique.
\item<1-> \textbf{Treillis complet} $L$ : ensemble ordonné $L = (Y, \leq)$ tel que $\bigvee S$ et $\bigwedge S$ existent pour tout sous-ensemble $S \subseteq Y$.

\medskip

\item<2-> \textbf{Sous-ensemble dirigé} $D \subseteq X$ : %pour toute paire $x,y$ d'éléments de $S$, il existe un majorant de $\{x, y\}$ dans $D$. 
$\forall x, y \in D, \exists z \in D, x \leq z \wedge y \leq z$.
\item<2-> \textbf{CPO} $P$ : ensemble ordonné $P = (X, \leq)$ tel que $\bigvee D = sup D$ existe pour tout sous-ensemble dirigé $D \subseteq X$.


\end{itemize}
\end{block}

%TODO oral : passer rapidement sur les défs.

\end{frame}

\begin{frame}

Exemples :


\begin{columns}
\begin{column}{0.5\linewidth}

\begin{figure}[ht]
\centering
\resizebox{0.5\linewidth}{!}
	{
\begin{tikzpicture}[]
   \node[circle, draw](bot) at (0,0) {$\bot$};
   \node[circle, draw](x1) at (-1.25,0.9) {$x_1$};
   \node[circle, draw](x3) at (-1.25,2.25) {$x_3$};
   \node[circle, draw](x2) at (1.25,1.5) {$x_2$};
   %\node[circle, draw, blue](top) at (0,3) {$x_4$};

   %\node[text = blue] (D) at (-3,2) {$D \subseteq X$};
   %\node[text = red] (vus) at (-4,1) {éléments traités};
   %\node[text = black] (cand) at (-2.5,-0.9) {candidats};

%   \draw (0) edge [in=60,out=120,loop] (0);
   \draw (bot) edge[->] node[left, below left] {$\leq$} (x1);
   \draw (bot) edge[->] node[left, below right] {$\leq$} (x2);
   \draw (x1) edge[->] node[left, right] {$\leq$} (x3);

   \draw[->] (x1) -- (x3);
\end{tikzpicture}
	}
\caption{CPO}
\end{figure}

\end{column}	
\begin{column}{0.5\linewidth}

%\vspace{1.4cm}

\begin{figure}
\centering
\resizebox{0.5\linewidth}{!}
	{
\begin{tikzpicture}[]
\node[circle, draw](bot2) at (6,0) {$\bot$};
   \node[circle, draw](x12) at (4.75,0.9) {$x_1$};
   \node[circle, draw](x32) at (4.9,2.25) {$x_3$};
   \node[circle, draw](x22) at (7.25,1.5) {$x_2$};
   \node[circle, draw](top2) at (6,3) {$x_4$};

%   \draw (0) edge [in=60,out=120,loop] (0);
   \draw[->] (bot2) -- (x12);
   \draw[->] (bot2) -- (x22);
   \draw[->] (x12) -- (x32);
   \draw[->] (x32) -- (top2);
   \draw[->] (x22) -- (top2);

   %\node[text = black] (lst1) at (6,-1) {$[x_1]$};
\end{tikzpicture}
	}
\caption{Treillis complet}
\label{ce}
\end{figure}

%TODO oral : donner l'exemple de {x1,x2} qui n'est pas dirigé et n'a pas de sup : ok CPO mais pas treillis. Pas le cas pour la deuxième figure.
%TODO oral : mentionner que tout ordre partiel fini avec un bottom est un CPO de toute façon (pour plus tard).

\end{column}
\end{columns}







%(F progressive : $\forall x \in X, x \leq F(x)$.) 

%TODO oral : expliquer que c'est aussi un treillis, et que ce n'est même plus un CPO si on enlève top (prendre l'ensemble des entiers positifs). En profiter pour dire qu'on va étudier les fonctions sur les CPOs et leurs points fixes, et expliquer un peu. Plus expliquer le contre exemple.

\end{frame}

\subsection{Les théorèmes de point fixe}
\begin{frame}[fragile]
\frametitle{Les théorèmes de point fixe}

%TODO oral : mentionner Knaster Tarski
\only<1,2,4>{
\textbf{Point fixe} $x \in X$ : $F(x) = x$. F \textbf{progressive} : $\forall x \in X, x \leq F(x)$.

F \textbf{monotone} : $\forall x, y \in X, x \leq y \implies F(x) \leq F(y)$.


\begin{alertblock}{théorème I (simplifié)}
Soient $P$ un CPO et $F : P \rightarrow P$ une fonction continue sur $P$. Alors $F$ a un plus petit point fixe donné par $\alpha := \bigvee_{n \geq 0}F^n(\bot)$.
\end{alertblock}
}

\only<1,2,4>{
\visible<2,4>{
\begin{alertblock}{théorème II : Pataraia}
Soient $P$ un CPO et $F : P \rightarrow P$ une fonction monotone. Alors $F$ a un plus petit point fixe.
\end{alertblock} %TODO oral : préciser que le code Coq s'éloigne de la preuve proposée par le bouquin, pour faire un hameçon à question. Réviser la méthode de voir P0 comme un point fixe minimal par Knaster Tarski, et pourquoi ça sert (plus).
}
}

\begin{onlyenv}<3>
Preuve de Pataraia : en deux principales étapes.
\begin{itemize}
\item Montrer que les fonctions monotones et progressives sur un même CPO ont un point fixe (commun).
\item Restreindre $F$ monotone à un sous-CPO \code{PF} sur lequel elle est progressive.
\end{itemize}

\medskip

Concrètement :

\begin{lstlisting}[frame=single, language = Coq, basicstyle=\scriptsize]
 Definition PF x := forall Y, Invariant F Y -> is_subCPO Y 
 	-> Y x.  (*P0 is the smallest F-invariant subCPO*)
 
 Definition H_sup := (sup (CPO := CPO_mon PF) 
                          Increasing_functions).
                          
 Definition fixpoint_II := (H_sup bot).
\end{lstlisting}

\end{onlyenv}

\only<1,2,4>{
\visible<4>{
\begin{alertblock}{théorème III : Bourbaki-Witt}
Soient $P$ un CPO et $F : P \rightarrow P$ une fonction progressive. Alors $F$ a un point fixe.

\end{alertblock}
}
}

%TODO oral : Ici, parler aussi en même temps des idées de preuve et des subtilités que ça implique en Coq.
%(y passer un peu de temps ! LONG !)


\end{frame}


\begin{frame}

Théorème de Bourbaki-Witt : existence de point fixe sans garanti de minimalité.

%Exemple :

\begin{figure}[ht]
\centering
\resizebox{0.8\linewidth}{!}
	{
\begin{tikzpicture}[]
   \node[circle, draw](bot) at (0,0) {$\bot$};
   \node[circle, draw](0) at (0,1.5) {$0$};
   \node[circle, draw](1) at (2,1.5) {$1$};
   \node[circle, draw](2) at (4,1.5) {$2$};
   \node[](+) at (6,1.5) {$...$};
   \node[circle, inner sep=2pt, draw](-1) at (-2,1.5) {$-1$};
   \node[circle, inner sep=2pt, draw](-2) at (-4,1.5) {$-2$};
   \node[](-) at (-6,1.5) {$...$};
   \node[circle, draw](top) at (0,3) {$\top$};

%   \draw (0) edge [in=60,out=120,loop] (0);
   \draw[->] (bot) -- (0);
   \draw[->] (0) -- (top);
   \draw[->] (bot) -- (1);
   \draw[->] (1) -- (top);
   \draw[->] (bot) -- (2);
   \draw[->] (2) -- (top);
   \draw[->] (bot) -- (-1);
   \draw[->] (-1) -- (top);
   \draw[->] (bot) -- (-2);
   \draw[->] (-2) -- (top);

   \draw[->, dashed] (-5.5,1.5) -- (-2);
   \draw[->] (-2) -- (-1);
   \draw[->] (-1) -- (0);
   \draw[->] (0) -- (1);
   \draw[->] (1) -- (2);
   \draw[->, dashed] (2) -- (5.5, 1.5);



%   \node[circle, draw](6) at (4,1) {6};
   \draw (top) edge[draw = blue, in=70,out=110,loop] node[text = blue, above] {F} ();
   \draw (2) edge[draw = blue, in=70,out=110,loop] node[text = blue, above] {F} ();
   \draw (1) edge[draw = blue, in=70,out=110,loop] node[text = blue, above] {F} ();
   \draw (0) edge[draw = blue, in=70,out=110,loop] (); %node[text = blue, right]  {F} ();
   \node[text = blue] (-) at (0.5,2) {$F$};
   \draw (-1) edge[draw = blue, in=70,out=110,loop] node[text = blue, above] {F} ();
   \draw (-2) edge[draw = blue, in=70,out=110,loop] node[text = blue, above] {F} ();
   \draw (bot) edge[->, draw = blue, bend left] node[left, text = blue] {F} (0);

%   \node[circle, draw](3) at (6,1) {3};
%   \node[circle, draw](5) at (6,0) {5};
   %\draw (2) edge [in=60,out=120,loop] (2);
\end{tikzpicture}
	}
\caption{CPO muni d'une fonction F progressive sans point fixe minimal}
\label{ce}
\end{figure}

%(F progressive : $\forall x \in X, x \leq F(x)$.) 


\end{frame}

\section{La formalisation en Coq}

\subsection{Les enjeux et problématiques de la formalisation}

\begin{frame}[fragile]
\frametitle{Les enjeux et problématiques de la formalisation}

\begin{block}{Critères et objectifs}
\begin{itemize}
\item \textbf{Minimalité} des axiomes requis
\item<2-> \textbf{Calculabilité} (problème avec $ \alpha = \bigvee_{n \geq 0} F^n(\bot)$)
\item<3-> \textbf{Universalité} : englober toutes les structures mathématiques de CPO et treillis
\end{itemize}
\end{block}

%\begin{itemize}
%\item 
\visible<3->{
\medskip

Comment représenter un sous-ensemble $S \subseteq X$ ? 

$\mathbb{1}_S$ \code{: X -> bool} ~~ ou ~~ $... \in S $ \code{: X -> Prop}
%\end{itemize}
}
%\medskip

%\pause
%\visible<2->{
%\begin{figure}
%\begin{lstlisting}[frame=single, language = Coq, basicstyle=\scriptsize]
%Axiom classic : forall P:Prop, P \/ ~ P.

%Axiom functional_extensionality : forall {A B : Type},
%  forall (f g : A -> B),
%  (forall x, f x = g x) -> f = g.
%\end{lstlisting}
%}
%\caption{exemple d'entrée}
%\label{ex}
%\end{figure}

%TODO oral : passer du temps à bien expliquer les enjeux de chaque. Si besoin pendant les répétitions, ajouter du contenu dans cette slide ou les suivantes (code ?)

\end{frame}


\subsection{Bibliothèque finale proposée}

%\subsection{Structure de valeurs de vérités}
%\label{verites}

\begin{frame}[fragile]
\frametitle{Ensemble de valeurs de vérité}


\only<1> {Pour réconcilier bool et Prop : paramétrisation par un domaine de valeurs de vérité.

\medskip}

%\scriptsize{
%\begin{figure}

\begin{onlyenv}<1>
\begin{lstlisting}[frame=single, language = Coq, basicstyle=\scriptsize]
Class B_param := { B : Type;
  (* Operations basiques sur B *)
  is_true : B -> Prop;
  BAnd : B -> B -> B;
  BAnd_spec : forall b1 b2, is_true b1 /\ is_true b2 
  	<-> is_true (BAnd b1 b2);
  [...] (* False, Or, Impl, ... *)
  }.
\end{lstlisting}

\begin{lstlisting}[frame=single, language = Coq, basicstyle=\scriptsize]
Class B_CL (B : B_param) P' := {
    Sup: (X -> B) -> X;
    sup_spec: forall D z, 
    			  (sup D <= z <-> forall y, D y -> y <= z);
  }.
\end{lstlisting}


\end{onlyenv}
%\caption{exemple d'entrée}
%\label{ex}
%\end{figure}
%}

\begin{onlyenv}<2->
\begin{lstlisting}[frame=single, language = Coq, basicstyle=\scriptsize]
Class B_param := { B : Type;
  (* Operations basiques sur B *)
  is_true : B -> Prop;
  BAnd : B -> B -> B;
  [...] (* False, Or, Impl *)

  (* Forall and Exists :*)
  K : Type -> Type;
  BForall : forall V, K V -> (V -> B) -> B (* /!\ *)
  [...]  (*specs et proprietes sur K *) }.
\end{lstlisting}
\end{onlyenv}

\only<2>{
\code{B = Prop} -> K prend tous les types;

\code{B = bool} -> K sélectionne les types finis. %Autre ? 

%Logique de Łukasiewicz ?
}

%TODO oral : raconter que seul bool et Prop sont pertinents avec cette formalisation de B, que les autres tentatives n'ont pas été concluantes.

\end{frame}


%
%\begin{frame}[fragile]
%\frametitle{Structures ordonnées}
%\subsection{Structure d'ordre, de CPO et de treillis}



%\begin{lstlisting}[frame=single, language = Coq, basicstyle=\scriptsize]
%Definition Directed {X} `(leq : X -> X -> B) (D : X -> B) 
%	: Prop := forall x y, 
%	D x -> D y -> exists z, D z /\ x <= z /\ y <= z.


%Definition directed_set `(leq : X -> X -> B) :=
%	{ Dbody : (X -> B) | (Directed leq Dbody) }.


%Class B_CPO `(P' : B_PO) := {
%    sup: directed_set leq -> X;
%    sup_spec: forall D z, 
%    			  (sup D <= z <-> forall (y:X), D y -> y <= z);
%  }.


%Sup: (X -> B) -> X;
%\end{lstlisting}

%TODO oral : s'il y a le temps ici, un peu de blabla sur les tentatives précédentes de formalisation, avec un sup total, avec un sup propositionnel.

%\end{frame}









%\subsection{Formalisations alternatives} --> enlevé pour gagner du temps de présentation sur le reste.

%\subsection{Fonction sup propositionnelle}
%\subsection{Première paramétrisation : Forall dépendants de l'ensemble support de l'ordre partiel}
%\label{param}

%\begin{frame}
%\frametitle{Formalisations alternatives}%ou un titre dans le genre

%TODO

%\end{frame}







\section{Le cas des CPOs finis}

\subsection{Représentation d'un ensemble fini}
\label{finis}

\begin{frame}[fragile]
\frametitle{Représentation et résultat principal}

Définition d'un ensemble fini en Coq :

\begin{lstlisting}[frame=single, language = Coq, basicstyle=\scriptsize]
Record fin X := {
  eq_dec : forall (a b : X), {a = b} + {a $\neq$ b};
  el : list X;
  all_el : forall a, List.In a el
  }.
\end{lstlisting}

\begin{alertblock}{Proposition}
Tout ensemble fini muni d'un ordre partiel et d'un élément bottom est un CPO.
\end{alertblock}


%TODO oral : ne pas trop insister, passer rapidement. Mettre surtout en avant le fait qu'on travaille avec des LISTES pour rendre compréhensible l'algorithme de recherche de sup et sa preuve.

\end{frame}


\subsection{Ordre partiel fini}
\label{ordrepartiel}

%\subsubsection{Algorithme utilisé pour obtenir le sup}

\begin{frame}
\frametitle{Algorithme de recherche d'éléments maximaux}

\only<1>{
\begin{figure}[ht]
\centering
\resizebox{0.4\linewidth}{!}
	{
\hspace{-3cm}
\begin{tikzpicture}[]
   \node[circle, draw](bot) at (0,0) {$\bot$};
   \node[circle, draw, blue](x1) at (-1.25,0.9) {$x_1$};
   \node[circle, draw, blue](x3) at (-1.1,2.25) {$x_3$};
   \node[circle, draw, blue](x2) at (1.25,1.5) {$x_2$};
   \node[circle, draw, blue](top) at (0,3) {$x_4$};

   \node[text = blue] (D) at (-4,2) {$D \subseteq X$ (à traiter)};
   \node[text = red] (vus) at (-4,1) {éléments traités};
   \node[text = black] (cand) at (-4,-0.9) {candidats max};

%   \draw (0) edge [in=60,out=120,loop] (0);
   \draw[->] (bot) -- (x1);
   \draw[->] (bot) -- (x2);
   \draw[->] (x1) -- (x3);
   \draw[->] (x3) -- (top);
   \draw[->] (x2) -- (top);

   \node[text = black] (lst1) at (0,-1) {$[~~]$};

\end{tikzpicture}
	}
%\caption{Algorithme de recherche d'éléments maximaux}
\end{figure}

}

\only<2>{
\begin{figure}[ht]
\centering
\resizebox{0.4\linewidth}{!}
	{
\hspace{-3cm}
\begin{tikzpicture}[]
   \node[circle, draw](bot) at (0,0) {$\bot$};
   \node[circle, draw, red](x1) at (-1.25,0.9) {$x_1$};
   \node[circle, draw, blue](x3) at (-1.1,2.25) {$x_3$};
   \node[circle, draw, blue](x2) at (1.25,1.5) {$x_2$};
   \node[circle, draw, blue](top) at (0,3) {$x_4$};

   \node[text = blue] (D) at (-4,2) {à traiter};
   \node[text = red] (vus) at (-4,1) {éléments traités};
   \node[text = black] (cand) at (-4,-0.9) {candidats max};

%   \draw (0) edge [in=60,out=120,loop] (0);
   \draw[->] (bot) -- (x1);
   \draw[->] (bot) -- (x2);
   \draw[->] (x1) -- (x3);
   \draw[->] (x3) -- (top);
   \draw[->] (x2) -- (top);

   \node[text = black] (lst1) at (0,-1) {$[~ x_1 ~]$};

\end{tikzpicture}
	}
%\caption{Algorithme de recherche d'éléments maximaux}
\end{figure}
}

\only<3>{
\begin{figure}[ht]
\centering
\resizebox{0.4\linewidth}{!}
	{
\hspace{-3cm}
\begin{tikzpicture}[]
   \node[circle, draw](bot) at (0,0) {$\bot$};
   \node[circle, draw, red](x1) at (-1.25,0.9) {$x_1$};
   \node[circle, draw, blue](x3) at (-1.1,2.25) {$x_3$};
   \node[circle, draw, red](x2) at (1.25,1.5) {$x_2$};
   \node[circle, draw, blue](top) at (0,3) {$x_4$};

   \node[text = blue] (D) at (-4,2) {à traiter};
   \node[text = red] (vus) at (-4,1) {éléments traités};
   \node[text = black] (cand) at (-4,-0.9) {candidats max};

%   \draw (0) edge [in=60,out=120,loop] (0);
   \draw[->] (bot) -- (x1);
   \draw[->] (bot) -- (x2);
   \draw[->] (x1) -- (x3);
   \draw[->] (x3) -- (top);
   \draw[->] (x2) -- (top);

   \node[text = black] (lst1) at (0,-1) {$[~ x_1, x_2 ~]$};

\end{tikzpicture}
	}
%\caption{Algorithme de recherche d'éléments maximaux}
\end{figure}
}

\only<4>{
\begin{figure}[ht]
\centering
\resizebox{0.4\linewidth}{!}
	{
\hspace{-3cm}
\begin{tikzpicture}[]
   \node[circle, draw](bot) at (0,0) {$\bot$};
   \node[circle, draw, red](x1) at (-1.25,0.9) {$x_1$};
   \node[circle, draw, red](x3) at (-1.1,2.25) {$x_3$};
   \node[circle, draw, red](x2) at (1.25,1.5) {$x_2$};
   \node[circle, draw, blue](top) at (0,3) {$x_4$};

   \node[text = blue] (D) at (-4,2) {à traiter};
   \node[text = red] (vus) at (-4,1) {éléments traités};
   \node[text = black] (cand) at (-4,-0.9) {candidats max};

%   \draw (0) edge [in=60,out=120,loop] (0);
   \draw[->] (bot) -- (x1);
   \draw[->] (bot) -- (x2);
   \draw[->] (x1) -- (x3);
   \draw[->] (x3) -- (top);
   \draw[->] (x2) -- (top);

   \node[text = black] (lst1) at (0,-1) {$[~ x_3, x_2 ~]$};

\end{tikzpicture}
	}
%\caption{Algorithme de recherche d'éléments maximaux}
\end{figure}
}

\only<5>{
\begin{figure}[ht]
\centering
\resizebox{0.4\linewidth}{!}
	{
\hspace{-3cm}
\begin{tikzpicture}[]
   \node[circle, draw](bot) at (0,0) {$\bot$};
   \node[circle, draw, red](x1) at (-1.25,0.9) {$x_1$};
   \node[circle, draw, red](x3) at (-1.1,2.25) {$x_3$};
   \node[circle, draw, red](x2) at (1.25,1.5) {$x_2$};
   \node[circle, draw, red](top) at (0,3) {$x_4$};

   \node[text = blue] (D) at (-4,2) {à traiter};
   \node[text = red] (vus) at (-4,1) {éléments traités};
   \node[text = black] (cand) at (-4,-0.9) {candidats max};

%   \draw (0) edge [in=60,out=120,loop] (0);
   \draw[->] (bot) -- (x1);
   \draw[->] (bot) -- (x2);
   \draw[->] (x1) -- (x3);
   \draw[->] (x3) -- (top);
   \draw[->] (x2) -- (top);

   \node[text = black] (lst1) at (0,-1) {$[~ x_4 ~]$};

\end{tikzpicture}
	}
%\caption{Algorithme de recherche d'éléments maximaux}
\end{figure}
}

%TODO oral : bien préciser que l'on obtient les éléments maximaux, pas forcément un sup. Il faut aussi que l'ensemble soit dirigé pour ça. Cf preuve.
   
\end{frame}

\begin{frame}[fragile]
\frametitle{Preuve de correction}

\textbf{Théorème :} Tout ensemble fini muni d'un ordre partiel et d'un élément bottom est un CPO.

\medskip

Étapes de la preuve :

\begin{itemize}
\item[1.] les éléments de la liste sont deux à deux incomparables
\item[2.] les éléments appartiennent à $D$
\item[3.] la liste ne contient pas de dupliqués
\item[4.] tout élément de $D$ est inférieur à un des éléments de la liste 
\item[5.] si $D$ est dirigé, la liste finale contient au plus un élément
\end{itemize}

\medskip

Invariant principal (point 4.) :

\begin{lstlisting}[frame=single, language = Coq, basicstyle=\scriptsize]
Lemma main_invariant D : forall lst x, D x -> 
	not (In x lst) \/ 
	exists y, In y (elem_max D lst nil) /\ x <= y.
\end{lstlisting}

\end{frame}


%\section*{Conclusion}

\begin{frame}[fragile]
\frametitle{Conclusion}

\begin{block}{En résumé}
\begin{itemize}
\item[->] Conception d'une bibliothèque générale sur les ordres partiels (dont CPOs et thms de points fixes).
\end{itemize}
\end{block}

Exemple concret de calcul de point fixe minimal, sur le CPO à trois éléments $\{\bot, x_1, \top\}$. %: $\bot ~ \leq ~ x_1 ~ \leq ~ \top$.

\medskip

\begin{columns}

\begin{column}{0.5\linewidth}

\usetikzlibrary {arrows.meta,automata,positioning}
\begin{figure}[ht]
\centering
\resizebox{0.3\linewidth}{!}
	{
\begin{tikzpicture}[shorten >=1pt,node distance=2cm,on grid,>={Stealth[round]},
    every state/.style={draw=black!50,very thick}]

  \node[state]  (bot)                      {$\bot$};
  \node[state]          (x) [above =of bot] {$x$};
  \node[state]          (top) [above =of x] {$\top$};

  \path[->] (bot) edge              node [right]  {$\leq$} (x)
            (x)      edge              node [right]  {$\leq$} (top)
            (top) edge [loop above, draw = blue, thick] node[text = blue]   {F} ()
            (x) edge [loop right, draw = blue, thick] node[text = blue]  {F} ()
            (bot) edge [bend left, draw = blue, thick] node[text = blue, left]  {F} (top) ;
\end{tikzpicture}
	}
%\caption{Top de $P_F$ non minimal pour $F$ progressive}
\end{figure}

\end{column}

\begin{column}{0.5\linewidth}

\begin{lstlisting}[frame=single, language = Coq, basicstyle=\scriptsize]
set (xf := lfp_II F).
vm_compute in xf.

xf := x.
\end{lstlisting}

\end{column}
\end{columns}


%\centering
%Merci de votre attention.

%\bigskip

%Des questions ?

\end{frame}



\end{document}